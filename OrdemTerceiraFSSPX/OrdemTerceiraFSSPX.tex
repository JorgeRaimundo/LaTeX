%! Author = jorge
%! Date = 24/04/20

% Preamble
\documentclass[11pt]{article}
\usepackage[utf8]{inputenc}

\title{Ordem Terceira de São Pio X}
\author{Priorado português da FSSPX}
\date{Abril 2020}

% Document
\begin{document}

\begin{titlepage}
\maketitle
\end{titlepage}

\section{O nascimento da Ordem Terceira}\label{sec:o-nascimento-da-ordem-terceira}

\subsection{Os primeiros pedidos dos leigos}\label{subsec:os-primeiros-pedidos-dos-leigos}

Em 28 de Maio de 1971, em Ecône, na vigília de Pentecostes, alguns fiéis leigos apresentaram-se a Dom Lefebvre:

--- ``Monsenhor, perguntaram, o senhor não tem uma espécie de Ordem Terceira?
Será que os leigos não poderiam aproximar-se um pouco mais da sua obra?''

--- ``É verdade, está nos estatutos que \textit{`A Fraternidade também acolhe os agregados, padres ou leigos, que desejem colaborar com o objectivo da instituição e aproveitar as suas graças para a própria santificação'} (IV,4).''

--- ``Então, Monsenhor, pode considerar-nos os seus primeiros terciários.''

--- ``Bom\ldots Reflictam.
Ainda não pensei sobre esse assunto, a não ser essa alusão aos estatutos.
Deixem-me pensar um pouco.''

\subsection{Dez anos de reflexão}\label{subsec:dez-anos-de-reflexão}

Dom Lefebvre iria ``pensar'' durante dez anos.
Enquanto isso, desde 1973, o fundador da Fraternidade, até então ajudado na sua gestão pelos ecónomos da Congregação do Espírito Santo, o irmão Christian Winckler, em Friburgo, e pelo padre Marcel Muller, em Paris, reflectia sobre a dispensa dos seus dois benévolos e devotados ajudantes, conforme o desejo manifestado pelo seus superiores.
Desde então ele dizia: \textit{``uma ordem Terceira de leigos seria útil para tarefas desse género''}.
Mas a finalidade espiritual continua a ter prioridade: viver da \textit{``nossa espiritualidade do Santo Sacrifício da Missa e da imolação;
penetrar sempre cada vez mais nesse grande mistério da nossa fé, tesouro do coração de Jesus, fonte de todo a amor verdadeiro e inalterável''}.
Entretanto a Ordem Terceira só nasceria a 29 de Janeiro de 1981, data na qual o Conselho Geral da Fraternidade promulgou as regras redigidas pelo fundador no final de 1980.

\subsection{Os deveres dos Terciários}\label{subsec:os-deveres-dos-terciários}

A uma vida cristã ``de sacrifício e de corredenção'', os terciários devem acrescentar o apego à Tradição expressa pelo magistério infalível e pelo Catecismo do Concílio de Trento, à Vulgata, aos ensinamentos do Doutor Angélico e à Liturgia de sempre.

Os deveres dos terciários são muito exigentes?
Na verdade, não!
Bem equilibrados, eles não vão além do que é possível pedir a fiéis fervorosos, nada muito difícil.
Mas o ambiente em comum vence o individualismo, estimula o zelo, eleva o nível da caridade e da oblação ao mais alto grau possível.
A Ordem Terceira forma, assim, ao lado do priorado da Fraternidade, uma elite espiritual e fervorosa\footnote{Dom Tissier de Malerais, em \textit{``Mysterium Fidei''}, Boletim de Ligação dos Terciários, Abril-Maio 2003, n.º 24}.

\section{Estatutos da Ordem Terceira de São Pio X}\label{sec:estatutos-da-ordem-terceira-de-são-pio-x}

\subsection{Finalidade da fundação da Ordem Terceira}\label{subsec:finalidade-da-fundação-da-ordem-terceira}

A santificação pessoal dos membros da Ordem Terceira e das pessoas que estão sob a sua responsabilidade.

\subsection{Patrono da Ordem Terceira: o Papa São Pio X}\label{subsec:patrono-da-ordem-terceira:-o-papa-são-pio-x}

Buscar a santificação hoje é algo que se realiza num mundo que se opõe através de erros e de heresias subtis, introduzidos em todos os meios católicos sob o nome de modernismo.

Ora, o Papa Pio X foi beatificado precisamente por ter denunciado corajosamente esses erros modernos e ter dado exemplo de santidade na firmeza da doutrina, na pureza dos costumes e na devoção ao Sacrifício Eucarístico.

Por isso este Santo Papa é especialmente indicado para servir de modelo, na nossa época, às almas que desejam se santificar.

\subsection{Filiação da Ordem Terceira à Fraternidade Sacerdotal São Pio X}\label{subsec:filiação-da-ordem-terceira-à-fraternidade-sacerdotal-são-pio-x}

A Ordem Terceira é uma fundação da Fraternidade e por isso os capelães da Ordem Terceira são designados pelos Superiores de Distrito e aprovados pelo Superior Geral.

Os membros da Ordem Terceira participam das graças da Fraternidade adquiridas pelas orações e méritos dos seus membros.

\subsection{Membros da Ordem Terceira}\label{subsec:membros-da-ordem-terceira}

Todos os católicos, sacerdotes ou leigos que aceitam o espírito e os Estatutos da Ordem Terceira.
As crianças podem se inscrever, com o consentimento dos seus pais, a partir dos doze anos.

\subsection{Insígnias}\label{subsec:insígnias}

A medalha de São Pio X e uma cruz, entregues na cerimónia de compromisso na Ordem Terceira.

\subsection{O espírito da Ordem Terceira}\label{subsec:o-espírito-da-ordem-terceira}

É o mesmo que anima a Fraternidade Sacerdotal, isto é, o espírito da Igreja, a sua fé viva manifestada por toda a sua Tradição, o seu magistério infalível, expresso e exposto no Catecismo do Concílio de Trento, na Vulgata, no ensinamento do Doutor Angélico, na Liturgia de sempre.

Espírito de adesão à Igreja Romana, aos Papas, aos Bispos;
espírito de obediência às autoridades da Igreja segundo a sua fidelidade à finalidade própria do seu cargo, que não é outra senão a de difundir a fé católica e o Reino de Nosso Senhor.

Espírito de vigilância em relação a tudo o que possa corromper a fé.

Devoção terna e filial à Virgem Maria segundo o espírito de São Luís Maria Grignion de Montfort, a São José e a São Pio X\@.

Redescobrir a importância capital do Santo Sacrifício da Missa e do seu mistério, para nele encontrar o sentido da vida cristã, vida de sacrifício e de corredenção.

\subsection{Etapas para ingressar na Ordem Terceira}\label{subsec:etapas-para-ingressar-na-ordem-terceira}

\begin{enumerate}
  \item \textit{Inscrição} --- A solicitação é dirigida ao sacerdote encarregado da Ordem Terceira ou ao Superior do Distrito.
  O sacerdote envia em resposta uma folha na qual se pedem algumas informações.
  Depois, se a solicitação for aceite, o sacerdote envia ao aspirante uma cópia da ficha de inscrição.
  \item \textit{Postulantado de um ano} --- Durante o qual a fidelidade do postulante é examinada em dois aspectos: no cumprimento das suas obrigações e na adesão ao espírito da Ordem Terceira.
  \item \textit{Compromisso} --- O postulante durante uma cerimônia pronuncia o seu compromisso diante do padre delegado para isso.
  Recebe então a medalha, o crucifixo e a carta de membro da Ordem Terceira.
\end{enumerate}

\subsection{Obrigações}\label{subsec:obrigações}

\subsubsection{Obrigações pessoais}

Diárias:
\begin{itemize}
	\item Orações da manhã e da noite, que podem ser Prima e Completas, ou as orações do Livro dos Retiros.
	\item Recitação do Terço
	\item Assistência à Missa de sempre e comunhão se for possível ou, na sua falta, quinze minutos de oração.
\end{itemize}

Semanais:
\begin{itemize}
	\item Assistência à Missa de sempre e não ao Novus Ordo Missae, por causa do perigo de se adquirir um espírito protestante.
\end{itemize}

Cada quinze dias:
\begin{itemize}
	\item se possível, o sacramento da penitência, ou pelo menos uma vez por mês.
\end{itemize}

Cada dois anos:
\begin{itemize}
	\item um retiro.
\end{itemize}

\textbf{Avisos práticos}

\begin{itemize}
	\item Leituras recomendadas: os escritos doutrinais de São Pio X, o Catecismo do Concílio de Trento, o Novo Testamento, a Imitação de Cristo, a vida dos santos.
    Difundir as boas leituras.
	\item Jejum nas Quatro Têmporas, nas Vigílias, na Quarta-feira de Cinzas e na Sexta-feira Santa.
    Abstinência de carne nas Sextas-feiras da Quaresma e em todas as Sextas-feiras do ano.
	\item Abster-se da televisão, abster-se de qualquer leitura indecente, praticar a sobriedade.
\end{itemize}

\subsubsection{Obrigações familiares}

Para os que estão unidos pelo laço do matrimônio:
\begin{itemize}
	\item Observar, com espírito de submissão a Nosso Senhor, as leis do matrimônio para ter uma família numerosa.
    Renunciar por completo a qualquer acção positiva destinada a não ter filhos.
    \item Fazer do lar familiar um santuário consagrado aos Corações de Jesus e de Maria, onde se reza em família pelo menos a oração da noite e, se for possível, também o Terço;
    um santuário onde reina a vida litúrgica pela observância de Domingos e dos dias de Festa e onde se repudia tudo o que possa manchar a alma das crianças: televisão, revistas indecentes.
    \item Escolher colégios verdadeiramente educadores e tradicionais e contribuir para a sua fundação.
    \item Ser prudente na escolha das diversões e dos lugares de férias.
\end{itemize}

\subsubsection{Obrigações profissionais e sociais}

\begin{itemize}
	\item Seguir o exemplo da Sagrada Família e cumprir com os deveres de justiça e de caridade, seja como empregador ou como empregado.
    \item Promover e defender o Reinado social de Nosso Senhor Jesus Cristo na sociedade, combater o liberalismo e o modernismo, pestes dos tempos modernos que abrem as portas da Igreja ao inimigo.
\end{itemize}

\subsection{Organização}\label{subsec:organização}

O padre encarregado da Ordem Terceira no Distrito recebe as inscrições, designa os seus assistentes e convoca as reuniões do seu Conselho e as reuniões gerais com o fim de animar os seus membros para que permaneçam ativos e atentos na obra da sua santificação e na santificação dos demais.

Esse sacerdote organiza uma biblioteca com um bibliotecário encarregado de emprestar os livros aos postulantes e aos membros, publica um boletim que une os terceiros entre si e lhes comunica endereços, horários de reuniões, cerimônias religiosas, peregrinações, ordenações, profissões religiosas, etc.

Nas reuniões efectua-se uma coleta para as necessidades da Ordem Terceira.
Um tesoureiro encarrega-se de apresentar as contas nas reuniões do Conselho.
O Conselho da Ordem Terceira decide sobre o uso dos recursos disponíveis: despesas internas ou ajuda a um seminarista, a uma escola, a enfermos, à organização de retiros, etc.

\section{Perguntas e respostas sobre a Ordem Terceira da Fraternidade São Pio X}\label{sec:perguntas-e-respostas-sobre-a-ordem-terceira-da-fraternidade-são-pio-x}

\subsection{O que é a Ordem Terceira da Fraternidade São Pio X?}\label{subsec:o-que-é-a-ordem-terceira-da-fraternidade-são-pio-x?}

A Ordem Terceira é a quinta família da Fraternidade São Pio X\@.
O leitor provavelmente sabe que a primeira família são os padres e seminaristas.
Esta é a família mais importante da FSSPX, uma vez que esta é uma \textit{``sociedade sacerdotal de vida comum sem votos''}.
O sacerdócio é de facto a principal preocupação da Fraternidade.
A Fraternidade tem agora em todo o mundo cerca de 550 sacerdotes e cerca de 250 seminaristas.

A segunda família são as irmãs da Fraternidade São Pio X, que tem mais de 200 membros, incluindo professas, noviças e postulantes.
Elas são as auxiliares dos sacerdotes pela sua vida quotidiana de oração (Missa, Rosário, Ofício Divino, Adoração do Santíssimo Sacramento), pelo seu apostolado activo (sacristia, catecismo, visita aos doentes) e pelo seu trabalho práctico (cozinha, lavandaria, etc.).

A terceira família são os irmãos. 
Actualmente eles têm cerca de 50 membros, que se dedicam em todo o mundo a apoiar os sacerdotes.

A quarta família são os oblatos, pessoas que vivem em comum com os sacerdotes, seminaristas, irmãos e freiras, mas sem fazer os votos.
Eles permanecem leigos, mas são uma grande ajuda para as diferentes casas em que se encontram.

A quinta família é a Ordem Terceira, que foi fundada em 1980.

\subsection{Qual foi o motivo da fundação da Ordem Terceira?}\label{subsec:qual-foi-o-motivo-da-fundação-da-ordem-terceira?}

Durante vários anos, um grande número de católicos foi sugerindo a mesma idéia.
Até mesmo pedidos urgentes eram comuns.
Um exemplo de carta:

\begin{quote}\textit{Estimado Padre, \ldots\ frequentemente rezo o terço pelas intenções da Fraternidade São Pio X.
A este propósito, gostaria de saber se existe uma Ordem Terceira da Fraternidade São Pio X e quais são as suas exigências.
Eu gostaria de viver uma vida mais consagrada\ldots\ uma entrega maior de mim segundo o espírito da Fraternidade.
Isso é possível?
Por favor, me informe.}\end{quote}

Os padres, portanto, sentiram a necessidade de proporcionar aos fiéis um meio de viver os ideais evangélicos, de manter a Fé, a Esperança e a Caridade, no meio da tempestade sem precedentes na Igreja, e, além disso, de encontrar a proteção de uma sólida fortaleza espiritual.
Eles comunicaram a Dom Lefebvre as suas preocupações e desejos, e os apelos urgentes dos fiéis angustiados, abandonados sem defesa nas ruínas das estruturas da Igreja, que antigamente garantiam e sustentavam sua fé.
A fundação da Ordem Terceira foi a resposta de Dom Lefebvre.

\subsection{Qual é o propósito da Ordem Terceira?}\label{subsec:qual-é-o-propósito-da-ordem-terceira?}

A Ordem Terceira da FSSPX é uma \textit{``Ordem criada para proteger as almas que vivem no mundo numa escola de santidade''}.
A santificação de indivíduos e daqueles que estão sob a responsabilidade dos membros da Ordem Terceira, tal é o propósito da Ordem Terceira.
Como as antigas e tradicionais Ordens Terceiras (carmelita, dominicana, franciscana\ldots), a Ordem Terceira da FSSPX é um estado de vida no meio do caminho entre o claustro e o mundo, ou por outras palavras, é uma Ordem religiosa que irá penetrar nos lares católicos no meio do mundo.

É \textit{importante} para alguém que é capaz de cumprir as obrigações da Ordem Terceira decidir-se a se tornar um membro?

Certamente sim.
Veja esta história:

\begin{quote}São Pio X tinha uma profunda compreensão das necessidades da Igreja e, portanto, muitas vezes fez afirmações penetrantes.
Certo dia, estando no meio de um grupo de cardeais, o Santo Padre disse-lhes: \textit{``Qual é a coisa mais necessária neste tempo para salvar a sociedade?'' ``Construir escolas católicas''}, disse um deles.  \textit{``Não.'' ``Multiplicar o número de paróquias''}, respondeu o outro. \textit{``Também não.'' ``Aumentar o número de padres''}, disse um terceiro. \textit{``Não, não''}, respondeu o papa. \textit{``O que é mais necessário no nosso tempo é ter em cada paróquia um grupo de leigos que seja ao mesmo tempo virtuoso, esclarecido, determinado, e realmente apostólico''} \footnote{\textit{A Alma do Apostolado}, de Dom Jean-Baptiste Chautard}.\end{quote}

Para salvar o mundo, São Pio X contava com católicos fervorosos que se dedicassem ao apostolado pela palavra e pela ação, mas principalmente pelo exemplo.

Portanto, a coisa mais importante em cada uma das nossas capelas é ter um pequeno grupo de almas fervorosas praticando a vida interior seriamente.
Este corpo de cristãos de elite será como o fermento que levantará o nível espiritual de todo o priorado pela sua vida de oração, pela sua ação discreta mas eficazmente apostólica, mas acima de tudo pelo seu bom exemplo.
A Ordem Terceira fornece às almas sedentas de perfeição um conjunto de regras que podem ajudá-las a alcançar este objetivo e, assim, santificar sua família e toda a missão.
Por exemplo, se 10 pessoas permanecessem na capela depois da Missa em ação de graças (porque são membros da Ordem Terceira), tenho certeza de que dentro de poucos meses toda a paróquia chegaria a fazer a devida ação de graças!

\subsection{Qual é o espírito da Ordem Terceira?}\label{subsec:qual-é-o-espírito-da-ordem-terceira?}

Ele está centrado na devoção ao Santo Sacrifício da Missa, que não é outra coisa senão o sacrifício da Cruz renovado sobre o altar de forma incruenta.
Os membros da Ordem Terceira unem-se a Nosso Senhor, a Vítima Divina, que Se oferece a Si mesmo por amor ao seu Pai e pelas almas.
É aqui que eles encontram a força que precisam no difícil caminho para a santidade.
A devoção a Nossa Senhora das Dores, a São José e a São Pio X também estão presentes na alma do membro da Ordem Terceira.

\subsection{Quais são as principais obrigações de um membro da Ordem Terceira?}\label{subsec:quais-são-as-principais-obrigações-de-um-membro-da-ordem-terceira?}

Algumas obrigações podem ser chamadas negativas: por exemplo, renunciar habitualmente à televisão.
Por quê?
Não só porque a maioria dos programas são ocasiões próximas de pecado por suas imagens indecentes, mas também porque, mesmo quando não apresentam sexo ou violência, a televisão permanece essencialmente imbuída do espírito do mundo.
Este espírito é um espírito de amor próprio, de orgulho, de conforto, de auto-satisfação, de prazer terreno.
Este espírito é bastante oposto ao espírito de Nosso Senhor, que é um espírito de amor a Deus, de humildade, de sacrifício, da Cruz, da verdadeira alegria espiritual.
Rádio e jornais são meios suficientes para se manter informado das notícias.

\subsection{Mas existem outras obrigações positivas?}\label{subsec:mas-existem-outras-obrigações-positivas?}

Sim, por exemplo, a oração da manhã e da noite, o terço diariamente, a confissão uma vez por mês.
Porém, embora essas obrigações sejam comuns a todos os bons católicos, outras obrigações são próprias para os membros da Terceira Ordem: 15 minutos de oração mental a cada dia (ou a Missa diária, onde é possível) e um retiro de dois em dois anos.

\subsection{Como posso me tornar um membro da Ordem Terceira?}\label{subsec:como-posso-me-tornar-um-membro-da-ordem-terceira?}

Depois de ter lido e meditado sobre os Estatutos da Ordem Terceira, basta pedir a um dos padres do priorado o formulário de inscrição na Ordem Terceira e seguir as indicações que forem dadas.
Caro leitor, Deus está esperando a sua generosidade.
Deus tomou um corpo e uma alma para morrer por nós.
E o que faço por Ele?
\textit{``Vamos, portanto, amar a Deus, porque Deus nos amou primeiro''}.
Que Deus o abençoe!

\end{document}