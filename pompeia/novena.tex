\documentclass[10pt,twoside,a5paper]{article}

\usepackage{fontspec}
\usepackage{color}
\usepackage{lettrine}
\usepackage{microtype}
\usepackage{pifont}
\usepackage{hyperref}
\usepackage{hyphenat}
\usepackage{polyglossia}
%\usepackage{csquotes}
\usepackage[a5paper]{geometry}

%\setmainfont[Ligatures={TeX,Common,Contextual,Discretionary,Rare,Historic},Contextuals=Alternate]{EB Garamond}
%\setmainfont[Ligatures={TeX,Common,Contextual,Discretionary,Rare,Historic},Contextuals=Alternate]{Linux Libertine G}
%\setmainfont[Ligatures={TeX,Common,Contextual,Discretionary,Rare,Historic},Contextuals=Alternate]{EB Garamond}

\setmainfont[
Ligatures={TeX,Common,Contextual,Discretionary,Rare,Historic}
,Contextuals={Alternate, WordInitial, WordFinal, LineFinal, Inner, Swash}
%,Contextuals={Alternate, WordInitial, LineFinal, Inner, Swash}
]{EB Garamond}

\setmainlanguage{portuguese}

\setlength{\parskip}{1em}

\renewcommand{\LettrineFontHook}{\color{red}}

\newcommand{\cross}[1]{
	\textcolor{red}{\ding{64}}#1
	% \textcolor{red}{\maltese}#1
}

\newcommand{\versic}[1]{
	\textcolor{red}{\char"2123\ }#1
}

\newcommand{\response}[1]{
	\textcolor{red}{\char"211F\ }#1
}


\title{Novena de São Pio de Pietrelcina a Nossa Senhora do Rosário de Pompeia}
\author{Jorge Gomes Raimundo}

\begin{document}
	
	\hyphenation{nas-ceu des-ceu re-mis-são mi-nhas nos-sas mor-te nos-so ver-bo}
	
	\begin{center}\Large{\textsc{Novena de São Pio de Pietrelcina a Nossa Senhora do Rosário de Pompeia}}\bigskip\end{center}
	
	\section*{A Novena}
	
	A ``Novena do Rosário de 54 dias'' é uma ininterrupta série de Rosários em honra de Nossa Senhora, revelada à doente incurável Fortuna Agrelli por Nossa Senhora de Pompeia, em Nápoles, Itália, no ano de 1884.
	
	Por 13 meses Fortuna Agrelli sofria de terríveis dores e nem mesmo os médicos mais célebres conseguiam curá-la. Em 16 de Fevereiro de 1884, a menina e os seus pais começaram uma novena do Rosário. A Rainha do Santo Rosário a premiou com uma aparição a 3 de Março. Maria sentava-se sobre um alto trono, rodeado por numerosas figuras; trazia o Seu Divino Filho sobre o colo e na mão um Rosário. Nossa Senhora e o Menino Jesus estavam acompanhados por São Domingos e Santa Catarina de Sena. O trono estava decorado com flores, a beleza de Nossa Senhora era maravilhosa. A Santa Virgem disse:
	
	\setlength{\leftskip}{3cm}
	
	\textit{``Filha, tu me invocaste com vários títulos e sempre obteve os meus favores. Agora, posto que me invocas com este título que muito me agrada, `Rainha do Santo Rosário', não posso mais recusar o favor que me pedes, porque este nome é o mais precioso e querido por mim. Faz três novenas e obterás tudo''.}
	
	\setlength{\leftskip}{0pt}
	
	Mais uma vez Nossa Senhora lhe apareceu e disse:
	
	\setlength{\leftskip}{3cm}
	
	\textit{``Qualquer um que deseje obter favores de mim deveria fazer três novenas da oração do Rosário e três novenas em agradecimento''.}
	
	\setlength{\leftskip}{0pt}
	
	Padre Pio depositava uma fé inabalável nas ``Três Novenas'' a Nossa Senhora do Rosário de Pompeia, recomendando-as muitas vezes aos seus filhos espirituais nas cartas que lhes escrevia.
	
	A ``Novena a Nossa Senhora do Rosário de Pompeia'' elaborada pelo Bem Aventurado Bartolo Longo, em 1879, deve ser rezada três vezes em seguida, para implorar graças nos casos considerados mais desesperados, seguida de outras três em agradecimento.
	
	Padre Pio recomendava ainda que os que fizessem deveriam receber a ``Comunhão diária em honra de Nossa Senhora'' e rezar o Rosário inteiro, todos os dias da Novena.
	
	\section*{Novena de petição}
	
	\lettrine{Ó}\ Santa Catarina de Sena, minha Protetora e Mestra, vós que assistis do Céu aos vossos devotos quando rezam o Rosário de Maria, assisti-me neste momento, e dignai-vos rezar juntamente comigo esta novena à Rainha do Rosário, que estabeleceu o trono das suas graças no Vale de Pompeia, a fim de que, por vossa intercessão, eu obtenha a graça que desejo. Amém.
	
	\versic\ Ó DEUS, vinde em meu auxilio.
	
	\response\ SENHOR, apressai-Vos em me socorrer.
	
	(Glória)
	
	\lettrine{Ó}\ Virgem Imaculada e Rainha do Santo Rosário, vós, nestes tempos de fé extinta e impiedade triunfante, quiseste estabelecer a vossa sede de Rainha e Mãe sobre a antiga Pompeia, morada dos mortos pagãos. E, desse lugar onde eram adorados os ídolos e os demónios, vós, hoje como Mãe da divina graça, espalhais, por toda a parte, os tesouros das celestes misericórdias. Oh! Volvei, também sobre mim, ó Maria, desse trono em que reinais piedosamente, os vossos olhos misericordiosos, e tende compaixão de mim, que tanto necessito do vosso socorro. Mostrai-vos também a mim, como a tantos outros vos tendes mostrado, verdadeira MÃE DE MISERICÓRDIA, enquanto eu, de todo o coração vos saúdo e vos invoco, ó soberana Rainha do Sacratíssimo Rosário. Amém.
	
	(Salve Rainha)
	
	\lettrine{P}{rostrado} aos pés do vosso trono, ó grande e gloriosa Senhora, a minha alma vos venera no meio dos gemidos e das angústias que pesam intensamente sobre ela. Nestes temores e agitações em que me encontro, levanto cheio de confiança os olhos para vós, que vos dignastes eleger para vossa morada as terras de pobres e abandonados camponeses. E ali, em face da cidade e do anfiteatro dos prazeres mundanos onde reinam o silêncio e a ruína, vós, como RAINHA DAS VITÓRIAS, levantastes a vossa voz poderosa para chamar de todos os pontos da Itália e do mundo católico os vossos devotos e filhos para vos construírem um Templo. Ó Senhora, tende compaixão desta minha alma que jaz na miséria; tende piedade de mim que estou cheio de miséria e humilhações. Vós, que sois o extermínio dos demónios, defendei-me destes inimigos que me assaltam; Vós, que sois o SOCORRO DOS CRISTÃOS, tirai-me destas tribulações em que miseravelmente estou mergulhado. Vós, que sois a NOSSA VIDA, triunfai da morte que ameaça minha alma nestes perigos a que está exposta: restituí-me a paz, a tranquilidade, o amor e a saúde! Amém.
	
	(Salve Rainha)
	
	\lettrine{S}{abendo} que são numerosos aqueles que têm sido por vós beneficiados, unicamente porque recorreram a vós com Fé, sinto novo desejo e coragem de vos invocar em meu socorro. Vós já prometestes outrora a São Domingos que quem quisesse graças com o vosso Rosário as obteria, e eu, com o rosário na mão, vos chamo, ó Mãe, ao cumprimento das vossas maternais promessas. Ainda agora, nos nossos dias, operais contínuos prodígios, a fim de levar os vossos filhos a vos honrarem no Templo de Pompeia. Quereis, pois, enxugar as nossas lágrimas, suavizar os nossos trabalhos! E eu, com o coração nos lábios, com viva fé, vos chamo e vos invoco: ``Minha Mãe!… Minha querida Mãe… Minha mãe tão bela!… Minha mãe dulcíssima, ajudai-me…'' MÃE E RAINHA DO SACRATÍSSIMO ROSÁRIO DE POMPEIA, não tardeis mais em estender-me a vossa mão poderosa para me salvar, pois a demora, como vedes, me levaria à ruína! Amém.
	
	(Salve Rainha)
	
	\lettrine{E}\ a quem mais recorrerei eu, senão a vós, que sois o ALÍVIO DOS AFLITOS, o CONFORTO DOS ABANDONADOS, a CONSOLAÇÃO DOS INFELIZES? Oh! Eu confesso, a minha alma é miserável, está sobrecarregada de enormes pecados, merece arder no Inferno, indigna de receber graças… Mas não sois vós a ESPERANÇA DOS DESESPERADOS, a grande MEDIANEIRA entre homem e DEUS, a nossa poderosa ADVOGADA junto ao trono do Altíssimo, o REFÚGIO DOS PECADORES? Oh! Basta que digais uma palavra, em meu favor, ao Vosso Filho e Ele vos ouvirá. Pedi-lhe, pois, ó Mãe, esta graça de que tenho tanta necessidade.
	
	(Expõe-se a graça desejada)
	
	Somente vós podeis obtê-la para mim. Vós, que sois a minha única esperança, a minha consolação, a minha doçura, e toda a minha vida. Assim espero! Amém.
	
	(Salve Rainha)
	
	\lettrine{Ó}{\ Virgem} e Rainha do Sacratíssimo Rosário, vós que sois a Filha do PAI Celeste, a Mãe do Divino Filho, a Esposa do Espírito Septiforme. Vós, que tudo podeis junto à SANTÍSSIMA TRINDADE, deveis impetrar-me esta graça, que me é tão necessária, desde que não sirva de obstáculo à minha salvação eterna. (Expõe-se novamente a graça desejada) Eu vo-la rogo, pelo Coração de vosso adorável JESUS, pelos nove meses em que o trouxeste em vosso seio, pelos méritos dos sofrimentos de sua vida, pela sua cruel Paixão, pela Morte na Cruz, pelo seu Nome Santíssimo, pelo seu preciosíssimo Sangue. Peço-a, enfim, pelo vosso dulcíssimo Coração, pelo vosso nome glorioso, ó Maria que sois a ESTRELA DO MAR, a SENHORA PODEROSA, a PORTA DO CÉU, a MÃE DE TODAS AS GRAÇAS. Em vós confio, de vós tudo espero de bom. Vós me haveis de salvar! Amém.
	
	(Salve Rainha)
	
	\versic\ Tornai-me digno de vos louvar, ó Virgem Puríssima!
	
	\response\ Dai-me Força contra vossos inimigos!
	
	\versic\ Rogai por nós, Rainha do Sacratíssimo Rosário!
	
	\response\ Para que sejamos dignos das promessas de Cristo!
	
	Oremos:
	\lettrine{Ó}{ Deus}, cujo Filho Unigénito, por sua Vida, Morte e Ressurreição nos mereceu a graça da salvação eterna, concedei, nós Vo-lo suplicamos, que meditando estes mistérios do Sacratíssimo Rosário da Bem–aventurada Virgem Maria, imitemos o que eles contêm e obtenhamos o que prometem. Por Cristo, Nosso Senhor. Amém.
	
	(Acrescenta-se uma Ave-Maria pelo advogado Bartolo Longo.)
	
	Oração a São Domingos e a Santa Catarina de Sena para obter graças de Nossa Senhora do Rosário de Pompeia:
	
	\lettrine{Ó} Santo Sacerdote de Deus e glorioso Patriarca São Domingos, que fostes o amigo, o filho predileto e o confidente da Rainha Celeste, e tantos prodígios operastes por virtude do Santo Rosário; e vós, Santa Catarina de Sena, primeira filha desta Ordem e poderosa medianeira junto ao trono de Maria e junto ao Sacratíssimo Coração de Jesus, com quem trocastes o vosso coração; vós, meus diletos Santos, olhai para as minhas necessidades e tende compaixão do estado em que me vejo. Vós tivestes na terra o coração aberto a todas as misérias alheias, e a mão poderosa para mitigá-las; agora no Céu, não se diminuiu nem a vossa caridade nem o vosso poder. Pedi, oh! Pedi por mim, à Mãe do Rosário e ao seu Divino Filho, já que tenho grande confiança de que, por vosso intermédio, hei de conseguir a graça que tanto desejo. Amém.
	
	(3 Glórias)
	
	(1 Glória em honra de São Francisco de Assis e 1 em honra de São Tomás de Aquino, para obter o dom da pureza.)
	
	\section*{Novena de agradecimento}
	
	Depois de rezar a primeira novena e receber a graça, reza-se esta novena meditando os quinze mistérios do Rosário, como pediu Nossa Senhora. Diante da imagem da Santíssima Virgem do Rosário, acendam-se duas velas, se possível, e, ajoelhados, tendo nas mãos o Rosário, reza-se:
	
	\versic\ Ó DEUS, vinde em meu auxilio.
	
	\response\ SENHOR, apressai-Vos em me socorrer.
	
	(Glória)
	
	\lettrine{E}{is-me} novamente aos vossos pés, ó Imaculada Mãe de Jesus, que vos comprazeis em ser invocada como Rainha do Rosário do Vale de Pompeia.
	
	Com alegria no coração, e com a alma compenetrada da mais viva gratidão, torno a vir a vós, minha generosa benfeitora, minha doce Senhora, soberana do meu coração; a vós que vos mostrastes verdadeiramente ser minha Mãe, Mãe que tanto me ama. Eu estava aflito, e vós me ouvistes, estava triste e me consolastes, estava em angústia e me restituístes a paz. Dores e penas de morte assaltaram meu coração, e vós, ó Mãe, do vosso trono de Pompeia, com um olhar compassivo, me serenastes o ânimo.
	
	Quem se dirigiu a vós com confiança, e não foi ouvido? Oh! Se todos soubessem quanto sois bondosa, quanto sois compreensiva com quem sofre, então todos recorreriam a vós!… Sede sempre bendita, ó Virgem Soberana do Rosário de Pompeia, por mim e por toda a humanidade, pelos Santos Anjos, por toda a terra como sois no Céu. Amém.
	
	(Salve Rainha)
	
	\lettrine{M}{il} graças rendo a Deus e mil graças a vós, ó Mãe divina, pelos novos benefícios que por vossa bondade e misericórdia me foram concedidos. Que teria sido de mim se tivésseis repelido os meus suspiros e as minhas lágrimas? Por mim vos agradeçam os Anjos do Paraíso, os Santos Apóstolos, os Mártires e os Confessores. Por mim vos agradeçam tantas almas pecadoras, salvas por vosso intermédio, que agora gozam no Céu a visão de vossa imortal beleza. Quisera que, juntamente comigo, todas as criaturas vos amassem e que o mundo todo repetisse o eco das minhas ações de graças. Que poderia eu vos oferecer, ó Rainha de piedade e magnificência?
	
	A vida que me resta eu a consagro a vós, e a propagar por toda parte o vosso culto, ó Virgem do Rosário de Pompeia, por cujas preces a graça do Senhor me visitou. Promoverei a devoção do vosso Rosário, contarei a todos a misericórdia que me alcançastes, proclamarei sempre quanto fostes boa para comigo, de modo que até os indignos e pecadores como eu para vós se voltem com confiança.
	
	(Salve Rainha)
	
	\lettrine{C}{omo} vos chamarei eu, ó cândida Pomba de paz? Com que títulos vos invocarei, vós a quem os Santos Doutores chamaram Senhora de tudo criado por Deus, Porta da Vida, Templo de Deus, Raio de Luz, Glória dos Céus, Santíssima de todos os Santos, Mártir dos Mártires, Paraíso do Altíssimo?
	
	Vós sois a tesoureira das graças, a omnipotência suplicante, a própria misericórdia de Deus que desce sobre os infelizes. Mas sei também que é agradável ao vosso coração ser invocada como RAINHA DO ROSÁRIO DO VALE DE POMPEIA.
	
	E, assim chamando-vos, sinto a doçura do vosso místico nome, ó Rosa do Paraíso, transplantada no vale do pranto para suavizar os trabalhos dos degredados filhos de Eva; rubra Rosa de Caridade, cujo perfume é mais agradável que todos os aromas do Líbano, que com a fragrância da vossa suavidade celestial atraís ao vosso Vale os corações dos pecadores e os conduzis ao Coração de Deus. Vós sois a Rosa de eterno frescor que, regada pelos arroios das águas celestes, lançastes as vossas raízes sobre a terra ressequida por uma chuva de fogo. Rosa de intemerata beleza, que no lugar da desolação plantastes o Jardim das delícias do Senhor.
	
	Adorado seja Deus que tornou o vosso Nome tão admirável! Bendizei, ó povos, bendizei o Nome da Virgem do Rosário, pois toda a terra está cheia de sua misericórdia. Amém.
	
	(Salve Rainha)
	
 	\lettrine{N}{o} meio das tempestades que me tinham submergido, levantei os meus olhos a vós, nova Estrela de esperança, aparecida nos nossos dias sobre o vale das ruínas.
	
	Da profundeza das amarguras ergui as minhas súplicas a vós, ó Rainha do Rosário de Pompeia, e experimentei o poder deste título que vos é tão caro.
	
	Salve! Clamarei sempre, salve! ó Mãe de piedade, mar imenso de graças, oceano de bondade e compaixão.
	
	As vossas glórias do vosso Rosário, as recentes vitórias da vossa coroa, quem as cantará dignamente? Vós, ao mundo, que se desprende dos braços de Jesus para se entregar aos de Satanás, ensinastes a salvação naquele vale, onde o inimigo atirou as almas.
	
	Vós calcastes, triunfadora, os alicerces dos templos pagãos e sobre as ruínas da idolatria pusestes o escabelo do vosso domínio. Vós mudastes a região da morte em vale de ressurreição e de vida, e sobre a terra dominada por vosso inimigo fundastes a cidade do refúgio, onde acolheis os povos para salvá-los.
	
	Eis que os vosso filhos, espalhados pelo Mundo, ali vos ergueram um trono, como sinal dos vossos prodígios, como troféu da vossa misericórdia. Vós, desse trono, me chamastes também a mim, entre os filhos da vossa predileção; sobre mim, pecador, também repousou o olhar da vossa misericórdia.
	
	Sejam benditas eternamente as vossas obras, ó Senhora, sejam benditos todos os prodígios por vós operados no Vale da desolação e do extermínio. Amém.
	
	(Salve Rainha)
	
	\lettrine{R}{essoe} em todas as línguas vossa glória, ó Senhora, e a tarde transmita ao dia seguinte o concerto das vossas bênçãos. Todos os povos vos chamem bem-aventurada e bem-aventurada repercutam todas as plagas da terra, as mansões dos Céus. Três vezes bem-aventurada, também eu vos chamarei com os Anjos, com os Arcanjos, com os Principados; três vezes bem-aventurada com as angélicas Potestades, com as Virtudes dos Céus, com as Dominações supernas, bem-aventurada, vos aclamarei com os Tronos, com os Querubins e os Serafins.
	
	Ó minha Soberana Salvadora, não deixeis de volver os vossos olhos misericordiosos sobre esta família, sobre esta nação, sobre toda a Igreja Católica apostólica. Sobretudo, não me negueis a maior das graças, isto é, que a minha fragilidade nunca me separe de vós.
	
	\section*{Os quinze mistérios do Rosário}
	
	Segundo Padre Pio
	
	Mistérios Gozosos

	\begin{enumerate}
		\item \textit{Anunciação}: Ó Maria, cheia de graça, pela humildade que fez de vós a Mãe de Deus, obtende também para nós uma encarnação do Verbo na qual toda a Vontade divina se possa cumprir.
		
		\item \textit{Visita de Maria Santíssima a Isabel}: Ó Maria, Mãe Divina, dai-nos o fruto de vosso ventre para que, com o vosso exemplo, possamos conquistar o nosso próximo para Jesus, com uma caridade maravilhosa.
		
		\item \textit{Nascimento do Menino Jesus}: Ó Maria, mais terna das Mães, preenchei o nosso coração de ternura pelo vosso pequeno Jesus e dai-nos a paz prometida aos homens de boa vontade.
		
		\item \textit{Apresentação de Jesus no templo}: Ó Maria, resplandecente de beleza na real oferta do vosso Jesus ao templo, oferecei-nos totalmente a Deus em acto de perfeita obediência.
		
		\item \textit{Reencontro de Jesus no templo}: Ó Maria, eterna suavidade, conservai-nos Jesus no coração, mas, se desafortunadamente O perdermos, fazei com que logo O reencontremos.
	\end{enumerate}
	
	Mistérios Dolorosos
	
	\begin{enumerate}
		\item \textit{Agonia de Jesus no horto do Getsémani}: Ó Jesus agonizante no horto das oliveiras, infundi em nós a força de superar os abandonos e as desolações do coração e dai-nos a contrição pela ofensa a Deus.
		
		\item \textit{Flagelação de Jesus}: Ó Jesus adorado, fazei que cada gota do sangue que derramastes das vossas mãos feridas seja voz poderosa que nos atraia a Vós sem reservas, para que possamos oferecer-Vos todo o nosso ser.
		
		\item \textit{Coroação de espinhos}: Ó Jesus escarnecido, reprimi a vaidade da nossa imaginação, separai-nos do transitório e prendei-nos àquilo que dura para sempre! Pelo vosso Sacerdócio Santo, dai-nos sacerdotes santos.
		
		\item \textit{Jesus carrega a sua Cruz}: Ó Maria, Mãe sofredora, fazei que convosco sigamos Jesus que se fez fraco para dar-nos força, que caiu para que pudéssemos nos levantar. Que nada nos afaste da subida do nosso Calvário, para chegar ao cume e ali morrer com Jesus, assistidos por vós, Mãe amorosa!
		
		\item \textit{Morte de Jesus na cruz}: Ó Maria, Rainha dos Mártires, fazei com que morramos para nós mesmos a fim de que possamos viver e morrer com Jesus e para Jesus. Que a nossa separação da terra seja um perfeito ato de amor e de sofrimento, um ansiado suspiro do ``Encontro''.
	\end{enumerate}
	
	Mistérios Gloriosos
	
	\begin{enumerate}
		\item \textit{Ressurreição de Jesus}: Ó Eterna Amada, dai-nos a vossa humildade, para que, quando morrermos, Jesus possa nos dizer: Vem, querida alma, eu mesmo quero levantar-te porque te fizeste pequena!
		
		\item \textit{Ascensão de Jesus ao céu}: Flor da Trindade, guiai-nos ao puro amor e fazei-nos compreender que na terra só teremos de conhecer uma única tristeza: a de não ser santos.
		
		\item \textit{Descida do Espírito Santo}: Mãe do amor maravilhoso, vós que experimentastes todas as doçuras, acendei no nosso coração a chama sagrada que nos faça morrer de amor para atirar-nos no eterno abraço ao vosso lado e do nosso amado Pai.
		
		\item \textit{Assunção de Maria ao céu}: Mãe dulcíssima, enquanto nos alegramos pela vossa gloriosa subida ao céu, fazei com que também nós possamos subir em companhia de todas as almas de nossos irmãos.
		
		\item \textit{Coroação de Maria Santíssima}: Ó Rainha do Paraíso, que, acima dos anjos e dos santos, estais à direita de Jesus, a vós suspiramos neste vale de lágrimas. Protegei-nos e não nos abandoneis até nos verdes salvos no céu abençoando e cantando as misericórdias de Deus.
	\end{enumerate}
	
	Texto baseado em: \url{http://www.amoranossasenhora.com.br/novena-de-sao-pio-de-pietrelcina-a-nossa-senhora-do-rosario-de-pompeia/}
	
	
\end{document}
